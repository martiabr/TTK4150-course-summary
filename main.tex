%%%%%%%%%%%%%%%%%%%%%%%%%%%%%%%%%%%%%%%%%%%%%%%%%%%%%%%
% A template for Wiley article submissions to Networks.
% Developed by Overleaf.
%
% Modified by D. Shier  (13 Feb 2019)
%%
% Usage notes:
% % Use "num-refs" option for numerical citation and references style.
% % Use the attached wiley-networks.cls and abbrv_networks.bst files.

\documentclass[num-refs]{wiley-networks}

% Add additional packages here if required
\usepackage{listings}
\usepackage{array}
\usepackage{mathtools}
\usepackage{bm}    
\usepackage{amsmath}
\usepackage{amssymb}
\usepackage{verbatim}
\usepackage{wrapfig}
\usepackage{cleveref}
%\usepackage{parskip}
\usepackage{graphicx}
\usepackage{caption}
\usepackage{siunitx}
\usepackage{subcaption}
\usepackage{enumitem}
\usepackage{mathrsfs}
\usepackage{tcolorbox}
\usepackage{xcolor}
\usepackage{soul}

\DeclareMathOperator*{\argmax}{arg\,max}
\DeclareMathOperator*{\argmin}{arg\,min}
\def\Var{\textsf{Var}}
\def\Cov{\textsf{Cov}}
\setcounter{MaxMatrixCols}{20}
\colorlet{soulteal}{teal!30}

% Update article type if known
\papertype{TTK4150}

\title{TTK4150 Nonlinear Systems and Control}

% Use the \authfn to add symbols for additional footnotes and present addresses, if any. Usually start with 1 for notes about author contributions; then continuing with 2 etc if any author has a different present address.

\author[1\authfn{1}]{Martin Brandt}

% Include the name of the author that should appear in the running header
\runningauthor{TTK4150 Summary}

\begin{document}

\setlist[itemize]{parsep=0pt}
\setlist[enumerate]{parsep=0pt}
\renewcommand{\labelitemi}{$*$}
\sethlcolor{soulteal}

\maketitle

\begin{abstract}
Learning goals:
\begin{itemize}
\item Hello
\end{itemize}
\end{abstract}
\setlength{\parindent}{0pt}
\newpage

\tableofcontents

\newpage

\section{Test}

\appendix

\section{Linear methods}
\begin{definition}
    We define the p-norm as:
    \begin{equation}
    \|x\|_{p}=\left(\sum_{i=1}^{n}\left|x_{i}\right|^{p}\right)^{\frac{1}{p}}, \quad p \in[1, \infty]
    \end{equation}
\end{definition}
\begin{theorem}
    Schwarz' inequality:
    \begin{equation}
        |<x, y>| \leq\|x\| \cdot\|y\|
    \end{equation}
\end{theorem}
\begin{definition}
    $f: \mathbb{R}^{n} \rightarrow \mathbb{R}^{m}$, then the Jacobian is defined as:
    \begin{equation}
    \frac{\partial f}{\partial x} \triangleq\left[\begin{array}{ccc}{\frac{\partial f_{1}}{\partial x_{1}}} & {\cdots} & {\frac{\partial f_{1}}{\partial x_{n}}} \\ {\vdots} & {} & {\vdots} \\ {\frac{\partial f_{m}}{\partial x_{1}}} & {\cdots} & {\frac{\partial f_{m}}{\partial x_{n}}}\end{array}\right]
    \end{equation}
    Which in the scalar case $m=1$ is the gradient.
\end{definition}

%REFERENCES
%Note that references are placed in lexicographic order by author name. Multiple entries within a citation should appear in numerical order also, such as [2, 17, 19, 42].

%\bibliographystyle{plain}
%\bibliography{sample}

\end{document} 